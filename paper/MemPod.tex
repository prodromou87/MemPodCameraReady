
%%%%%%%%%%%%%%%%%%%%%%%%%%%%%%%%%%%%
% This is the template for submission to ISCA 2016
% The cls file is a modified from  'sig-alternate.cls'
%%%%%%%%%%%%%%%%%%%%%%%%%%%%%%%%%%%%

\documentclass{sig-alternate}
%\usepackage{mathptmx} % This is Times font

%\newcommand{\ignore}[1]{}
\usepackage{fancyhdr}
\usepackage[normalem]{ulem}
\usepackage[hyphens]{url}
\usepackage{hyperref}

\usepackage{tabularx}
\usepackage{ulem}
%\usepackage{program}


%%%%%%%%%%%%%%%%%%%%%%%%%%%%%%%%%%%%%%%%%%%%%%%%%%
% I need these so it compiles on my Ubuntu machine
% There's some issue with the algorithm2e package
\makeatletter
\newif\if@restonecol
\makeatother
\let\algorithm\relax
\let\endalgorithm\relax
%%%%%%%%%%%%%%%%%%%%%%%%%%%%%%%%%%%%%%%%%%%%%%%%%%
\usepackage[boxruled]{algorithm2e}


%\usepackage{algorithm2e}
%\usepackage{mathptmx} % This is Times font

%%%%%%%%%%%---SETME-----%%%%%%%%%%%%%
\newcommand{\microsubmissionnumber}{266}
%%%%%%%%%%%%%%%%%%%%%%%%%%%%%%%%%%%%

\fancypagestyle{firstpage}{
  \fancyhf{}
\setlength{\headheight}{50pt}
\renewcommand{\headrulewidth}{0pt}
  \fancyhead[C]{\normalsize{MICRO 2016 Submission
      \textbf{\#\microsubmissionnumber} -- Confidential Draft -- DO NOT DISTRIBUTE}}
  \pagenumbering{arabic}
}

%%commands
\usepackage{ifthen}
\usepackage{calc}
\usepackage{pifont}
\usepackage{color}
\usepackage{fancyhdr}

\newcounter{hours}
\newcounter{minutes}
\newcommand{\printtime}{%
  \setcounter{hours}{\time/60}%
  \setcounter{minutes}{\time-\value{hours}*60}%
  \thehours:\theminutes}

\newcommand{\myspacing}{1.62}
\newcommand{\linespacing}[1]{\renewcommand{\baselinestretch}{#1}\normalsize}

% Set timeofmake to 'true' to print the time the file was compiled
\newboolean{timeofmake}
\setboolean{timeofmake}{false}


\newcommand{\todo}[1]{{\color{red}\sf\bfseries #1}}
\newcommand{\dontinclude}[1]{ }
\newcommand{\myfootnote}[1]{\footnote{\renewcommand{\baselinestretch}{1.0}\scriptsize #1}}


\newcommand{\putsec}[2]{\vspace{-0.1in}\section{#2}\label{sec:#1}\vspace{-0.0in}}
\newcommand{\putssec}[2]{\vspace{-0.05in}\subsection{#2}\label{ssec:#1}\vspace{-0.0in}}
\newcommand{\putsssec}[2]{\vspace{0.0in}\subsubsection{#2}\label{sssec:#1}\vspace{0.0in}}
\newcommand{\putsssecX}[1]{\vspace{0.0in}\subsubsection*{#1}\vspace{0.0in}}

%\renewcommand{\figurename}{\renewcommand{\baselinestretch}{1.0}\vspace{-0.0in}\footnotesize Figure}
%\renewcommand{\tablename}{\renewcommand{\baselinestretch}{1.0}\vspace{-0.0in}\footnotesize TABLE}
 
\newcommand{\tabput}[3]{
\begin{table}[t]
\renewcommand{\baselinestretch}{0.95}
%\small %also for forcing a baselinestretch update
\begin{center}
{
#2
\vspace{-0.1in}
}
\caption{\footnotesize #3 \label{tab:#1}\vspace{-0.2in}}
\renewcommand{\baselinestretch}{\myspacing}\normalsize
\end{center}
\end{table}
}
  
\newcommand{\tabputW}[3]{
\begin{table*}[h]
\begin{center}
{
#2
}
\end{center}
\caption{\footnotesize #3 \label{tab:#1}}
\end{table*}
}

\newcommand{\figput}[4][1.0\linewidth]{
\begin{figure}[t]
\begin{minipage}{\linewidth}
\footnotesize %also for forcing a baselinestretch update
\begin{center}
\includegraphics[trim=#3, clip, width=#1]{plots/#2}
\end{center}
\vspace{-0.1in}
\caption{\footnotesize #4 \label{fig:#2} \vspace{-0.0in}}
\end{minipage}
\end{figure}
}


\newcommand{\figputW}[4][1.0\linewidth]{
\begin{figure*}
\begin{minipage}{\linewidth}
\footnotesize %also for forcing a baselinestretch update
\begin{center}
\includegraphics[trim=#3, clip, width=#1]{plots/#2}
\end{center}
\vspace{-0.1in}
\caption{\footnotesize #4 \label{fig:#2}}
\end{minipage}
\end{figure*}
}

\newcommand{\figputT}[3]{
\begin{figure}[t]
\begin{minipage}{\linewidth}
\footnotesize %also for forcing a baselinestretch update
\begin{center}
\includegraphics[trim=#2, clip, width=1.0\linewidth]{plots/#1}
\end{center}
\vspace{-0.1in}
\caption{\footnotesize #3 \label{fig:#1}}
\end{minipage}
\end{figure}
}

\newcommand{\figputJ}[4][1.0\linewidth]{
\begin{figure}[htb]
\begin{minipage}{\linewidth}
\footnotesize %also for forcing a baselinestretch update
\begin{center}
\includegraphics[trim=#3, clip, width=#1]{plots/#2}
\end{center}
\caption{\footnotesize #4 \label{fig:#2}}
\end{minipage}
\end{figure}
}

\newcommand{\figref}[1]{Figure~\ref{fig:#1}}
\newcommand{\tabref}[1]{Table~\ref{tab:#1}}
\newcommand{\secref}[1]{Section~\ref{sec:#1}}
\newcommand{\ssecref}[1]{Section~\ref{ssec:#1}}
\newcommand{\sssecref}[1]{Section~\ref{sssec:#1}}

\newcommand{\ignore}[1]{}



\newcommand{\TODO}[1]{\textcolor{red}{\todo{#1}}}
\renewcommand{\arraystretch}{1.5}

%%%%%%%%%%%---SETME-----%%%%%%%%%%%%%
\title {MemPod: A Clustered Architecture for Efficient and Scalable Migration in Flat Address Space Multi-level Memories}
\author{}
%%%%%%%%%%%%%%%%%%%%%%%%%%%%%%%%%%%%

\begin{document}
\maketitle
\thispagestyle{firstpage}
\pagestyle{plain}

\begin{abstract}

In the near future die-stacked DRAM will be present alongside off-chip memories in hybrid memory systems. A large body of recent research has explored possible uses and performance-improving mechanisms regarding such a memory configuration. The first approach is to use this new memory as a last level cache and it has been shown to be beneficial for system performance. Another approach is to use the on-chip memory as an extension of the off-chip one in a ``flat address space'' configuration, exposing more memory capacity to applications leading to performance improvement under capacity constrained workloads.

In this paper we propose MemPod, a scalable and efficient memory management mechanism for flat address space hybrid memories. MemPod monitors main memory activity and upon predefined intervals migrates the most frequently accessed memory pages to the faster on-chip memory. We propose a novel architectural organization and the use of ``Majority Element Algorithm'' for activity tracking. 

Our results with multi-programmed workloads show MemPod to improve Average Main Memory Time (AMMT) by up to 29\% and 11\% on average compared to the state-of-the-art, while our analysis shows that MemPod is the most viable option compared to the state of the art, as memories become faster in the future. MemPod's novel activity tracking approach leads to significant cost reduction ($\sim$6000x lower space requirements) and incredible improvement on future prediction accuracy (58\% average improvement) over traditionally used full counters. 

\end{abstract}

\noindent \textbf{Keywords:} Memory architecture, Die-stacked memory

% !TEX root = ../MemPod.tex
\section{Introduction}
\label{sec:Introduction}
  
%The memory wall problem has been known to impede system performance \cite{wulf-can95}. 3D stacked memory is considered as a viable solution and has been studied extensively. Placing memory stacks in the processor package was shown to provide significant improvements in terms of bandwidth and power consumption \cite{black-micro2013}. Processor manufacturers already began including 3D-stacked memories in their announced products \cite{KnightsLanding,NVIDIA,black-micro2013} and die-stacked memory standards have been developed \cite{jedec-wideio,JEDEC-HBM,pawlowski-hotchips2011}. The use of die-stacked memory will undeniably be part of many future systems. Products such as \TODO{Cite Fury-X, any other existing product (AMD?)} are already manufactured with stacks of memory in the same package as the processing cores.

%The current technology for incorporating stacked memory, as well as the current protocols allow up to 8GB per DRAM stack \cite{JEDEC-HBM-REVISED}. Commodity server systems often need {\em hundreds} of GB of memory. Consequently, at its current state, in-package memory cannot solely support today's memory requirements, leading to the emergence of \emph{hybrid memories}, usually with fast 3D-stacked memory and the traditionally used off-chip memory such as DDR4. Often, hybrid configurations with two memory technologies are called \emph{``Two-Level Memories''}, or \emph{``N-Level Memories'' (NLM)} for configurations with N memory technologies. It is not clear yet how this newly added memory is best utilized. Recent research proposed mechanisms to manage the 3D-stacked memory as a high-bandwidth last level cache, while other proposals attempt to manage this memory as an extension of the main memory. Each approach comes with its own challenges, benefits and drawbacks. 

Die-stacked memory will increasingly be part of future systems, 
attempting to alleviate 
the memory wall problem~\cite{wulf-can95}. Processor manufacturers are 
already announcing products featuring 3D-stacked memory \cite{KnightsLanding,NVIDIA,black-micro2013} \TODO{Cite AMD graphics cards} and memory standards have been developed~\cite{jedec-wideio,JEDEC-HBM,pawlowski-hotchips2011}. 
Currently, this technology is limited to 8GB per stack~\cite{JEDEC-HBM-REVISED}
which does not fully address the capacity demands of modern systems. 
Therefore, die-stacked memories are expected to coexist with larger, 
slower off-chip memories, such as DDR4, in a configuration often
referred to as ``Two-Level Memory'' (TLM) \cite{cameo,meswani-HPCA21}.  

%Recent proposals in the literature \cite{chou-micro2014,qureshi-micro2012} demonstrate that when stacked memory is used as a large high-bandwidth last level cache, we need to re-evaluate traditional SRAM cache optimizations. Tag placement and granularity must be carefully studied and double access (tag check and data retrieval) in order to serve one request must be avoided due to the high latency of accessing a DRAM structure. Despite the associated challenges, state-of-the-art proposals manage to achieve high performance improvement through elegant solutions. Such a management scheme offers transparent operation and does not require added hardware structures to support it. On the downside, the extra memory capacity is not available to the software and some space is wasted to store tags instead of useful information.

Stacked memory can be used as a large, high-bandwidth last level cache, or as part of main memory in a ``flat address space''. When used as a cache, recent research~\cite{chou-micro2014,qureshi-micro2012} demonstrates the need to re-evaluate traditional SRAM-based cache organizations. Tag placement and granularity needs to be re-evaluated, and we should avoid double memory accesses for tag 
check and data retrieval. Managing stacked memory as a cache is transparent 
to the software and improves the performance of latency-sensitive applications. However, capacity-sensitive applications do not gain significant improvements as the stacked memory's capacity is not utilized for additional storage.

%Instead of using stacked-memory as cache, some recent research \cite{sim-micro2014,meswani-HPCA21,CAMEO} proposed both hardware and hardware-supported OS dynamic memory management mechanisms to manage it as a \emph{``flat address space memory''}. Memory accesses are being monitored in hardware and the goal is to identify ``hot'' memory pages and migrate them into the fast memory, in order to improve performance. Such a configuration extends the exposed main memory capacity, serving capacity-constrained workloads better than a DRAM cache organization, while at the same time eliminating the issue of tag placement and retrieval. Additionally, a flat address space organization could be incorporated in a system without the presence of a dynamic memory management mechanism and it would still operate correctly and boost performance simply because of the presence of a fast memory region and static memory allocation from the OS. Like a DRAM cache, this memory organization is also transparent to the application programmer.

In a flat address space configuration, the capacity of stacked memory is 
available and can be allocated and used by applications. Dynamic memory 
managers proposed in the literature~\cite{sim-micro2014,meswani-HPCA21,CAMEO} 
monitor and profile memory accesses and attempt to transparently migrate 
frequently accessed pages to the fast portion of memory. While exposing
more memory to the system, profiling memory accesses and performing 
transparent migrations often come with power, performance, and space overheads. 

%Flat address space dynamic memory managers need to overcome significant challenges. The main drawback comes from the required book-keeping. Monitoring memory regions and keeping track of migrated pages in order to relay incoming requests transparently, often comes at very high storage and power consumption overheads. Unlike a DRAM cache, there is no backing store memory. As such, each migration is a swap of two pages in order to ensure that exactly one copy of each participant page exists in main memory. In the presence of a flat memory with two regions, one fast and one slow, it could be expected that the fast one is fully utilized before we start using the slow one. However, modern OS's assign memory addresses randomly for security reasons \TODO{Find and cite}. Identifying and migrating hot pages dynamically can result in an optimized page placement.

%\TODO{say a couple of lines about why THM and CAMEO are not good, dont need deep details just say that related work has some disadvantages and say what they are at a high level}

Software migration schemes \cite{meswani-HPCA21} have high performance 
overheads and operate at coarse intervals, and thus are slow to adapt to 
changes in application phases. Recently proposed hardware managed schemes \cite{sim-micro2014,CAMEO} operate at finer granularity than software by either using simple, cache-like demand-driven migration or using a centralized management scheme. The former does not consider hotness of data, whereas the latter will not scale to large memories due to its centralized approach.

This paper introduces MemPod, a dynamic memory manager for flat address space memory configurations that is area efficient and high performance.
It scales particularly well to future technologies with larger capacities and
higher memory technology performance differentials. 
MemPod's novel microarchitectural design clusters existing memory controllers into memory ``Pods". Each Pod operates independently and in parallel allowing for better scalability and integration to future systems with larger and faster memories. For MemPod's activity tracking requirements we incorporate the \emph{``Majority Element Algorithm'' (MEA)} heuristic \cite{karp-mea,charikar-mea}, originally proposed for database management and big data analytics. Our evaluation shows MEA to be capable of high prediction accuracy with very low hardware overhead. This algorithm has 
not previously been proposed for activity tracking in hardware.

Our evaluation results with homogeneous and mixed 8-core multi-programmed workloads show MemPod to outperform the current state-of-the-art by 11\% on average and up to 29\% in terms of Average Main Memory Access Time (i.e. the average time a request spends waiting for main memory). Modeling future memory configurations, our results show MemPod to be the most scalable mechanism as memory technology improves. The use of MEA activity tracking requires $\sim0.01\%$ of the storage space required by the Full Counters (FC) approach used in previous research studies which uses one access counter per memory page or region, while at the same time achieving more accurate prediction
of future hot pages.
\remark{A.P.: Added a small explanation of FC after Nuwan's comment}
\remark{took out the number since our definition of accuracy was very arbitrary}

\remark{It's very hard to claim this as a contribution.  It's just a list of
properties anyone could come up with.  So I de-emphasized a bit and removed
as a claimed contribrution.}

We identify the fundamental building blocks of any flat address space dynamic 
memory management mechanism, and describe the solution for each block in 
prior proposed systems and MemPod, along with their various tradeoffs. 
%System designers can choose any solution for each of these elements and create a new management scheme in a plug-and-play fashion. Each building block is associated with its corresponding trade-off. In Section \ref{sec:Architecture} we provide a description of each block, along with a side-by-side comparison of MemPod's approach and what the state-of-the-art mechanisms propose.

\remark{without the ``building block'' as a contribution, this list
seems pretty sparse, so I removed the contributions list.}
\ignore{
The contributions of this paper are:

\begin{itemize}
\item Novel activity tracking algorithm (Section \ref{sec:MEA}).
\item Novel clustered microarchitecture (Section \ref{sec:Architecture}).
\item Breakdown of the basic building blocks of \emph{any} flat address space dynamic memory management mechanism. 
\item Evaluation of MemPod's effectiveness and its sensitivity to design parameters (Section \ref{sec:Results}).
\end{itemize}
}

\remark{Now this section ends abruptly.  Should end with the usual
``This paper is organized as follows...'' or a quick summary of numerical
results/gains.}

%In Section \ref{sec:Background} of this paper we present background and related work regarding memory management schemes. Section \ref{sec:MEA} presents our novel activity tracking in detail along with an in-depth evaluation of its capabilities compared to traditionally used mechanisms. Section \ref{sec:Architecture} gives a detailed overview of MemPod's architecture, presented side-by-side with the state-of-the-art mechanisms we compare against. Our architecture section is organized based on the fundamental basic blocks of management mechanisms in order to present the management problem in its entirety. Section \ref{sec:Results} presents our experimental methodology and evaluation results and finally, Section \ref{sec:Conclusions} concludes the paper.

% !TEX root = ../MemPod.tex

\section{Background \& Related Work}
\label{sec:Background}
Description of memory organization

Differences between DDR, HBM, PCM, on-chip/off-chip, benefits and drawbacks of each memory type.

Present the experiment where we identify the optimal NLM organization

%% !TEX root = ../HPCA2016.tex
\section{Related Work}
\label{sec:Related}

Present the papers in a structured way. Organize them in categories and discuss a category at a time. E.g. HW-assisted, SW-assisted, Hybrid. Separate category for irrelevant papers that worked as motivation.

List all the papers we think are relevant. Dicuss each one briefly.

List the elements that make our work better than the rest.

% !TEX root = ../MemPod.tex

\section{Predicting Hot Regions}
\label{sec:MEA}

%Memory management mechanisms need the ability to monitor and profile accesses to memory in order to identify hot regions and migrate them. Traditionally, a full set of counters is used, with one counter per physical page -- or region depending on the mechanism's granularity -- in order to accurately keep count of all accesses to all regions. Periodically these counters must be sorted in order to identify the regions with the highest counts. The identified regions will serve as a ``prediction'' for the next interval (i.e. the hottest page of the current interval will be amongst the hottest pages of the following intervals).

Migration mechanisms predict \textit{future} hot pages to migrate them into fast memory. Prediction accuracy is critical to high performance, as each migration must be amortized by many future accesses to justify the cost of migration.
A commonly used practice is to identify the hot regions within an interval 
and assume that those regions will be hot in the next interval. To accurately identify the hottest regions some mechanisms use an access counter per region. 
At the end of each interval the counters are sorted to identify the highest 
ranked (i.e., most accessed) regions. However, application phase changes could render this approach unsuccessful. Additionally, the number of necessary counters increases linearly as memory capacities grow.

To address the above limitations, we adopt a technique based on the ``Majority Element Algorithm'' $(MEA)$. MEA was originally proposed by Karp et al. \cite{karp-mea} and was studied in-depth by Charikar et al. \cite{charikar-mea} for database management and big data analytics. This heuristic has formally been proven to correctly identify the $K$ most frequently occuring elements of a set, when each of those elements appears more than $N \over K+1$ (i.e. has majority), where $N$ is the number of elements in that set. 

%MemPod uses the ``Majority Element Algorithm'' $(MEA)$ for its activity tracking needs. MEA attempts to identify the \textit{majority} elements in a set. For example in an array with integers, MEA can be used to identify the \textit{K majority numbers}. Majority elements are the most frequently occurring elements, as long as they exist more than $N \over K+1$ times in the information stream (i.e. they have majority), where N is the number of elements in the array.

%The MEA algorithm was originally proposed in \cite{karp-mea} and studied in-depth in \cite{charikar-mea} as a heuristic capable of efficiently identifying majority elements in a stream. This algorithm is formally proven to be 100\% accurate, as long as the K most frequently occurring elements it must identify have majority. With complexity $O(N)$ it could be an ideal candidate for real-time streams of information, such as a stream of memory requests.


\setlength{\textfloatsep}{5pt}
\begin{algorithm}[t]
\centering
% \small
% \DontPrintSemicolon
% %\dontprintsemicolon
% \;
% \PrintSemicolon
% %\printsemicolon
% 
% \KwIn{X: Set of N elements}
% \KwIn{K: Number of elements to output}
% \KwData{T: Map structure with K entries}
% \KwResult{Set of K majority elements}
% \DontPrintSemicolon
% %\dontprintsemicolon
% \;
% Initialization: $T \leftarrow \emptyset$\; 
% \;
% \PrintSemicolon
% %\printsemicolon
% \ForEach{$i \in X$} {
% 	\uIf{$i \in T$}{
%		$T[i] \leftarrow T[i] + 1$\;
%	}
%	\uElseIf{$|T| < K - 1$}{
%		$T[i] = 1$\;
%	}
%	\Else{
%		\ForAll{$j \in T$}{
%			$T[j] \leftarrow T[j] - 1$\;
%			\lIf{$T[j] == 0$}{$T \leftarrow T \setminus {j}$}
%		}
%	} 
% }
% 
 \includegraphics[scale=0.8]{figures/mea_algorithm.pdf}
 %\caption{Majority Element Algorithm}
 \label{alg:mea}
\end{algorithm}

%\begin{algorithm}
%	\includegraphics[width=0.45\textwidth]{figures/mea_algorithm.pdf}
%	\caption{TEST}
%	\label{alg:mea}
%\end{algorithm}

MEA is presented in Algorithm \ref{alg:mea} as applied to an array of integers X. A map structure T maps K element IDs (in our integer array example, IDs are the integers' values) to K counters. Looping through the array, if the next integer exists in the map, its counter is incremented by 1. Otherwise, if there's enough room in the map a new entry is added with a count of 1. If the number does not exist in the map and all K counters are occupied, the algorithm subtracts 1 from every counter, removes the entries with a counter value of 0 and proceeds to the next integer. Once the entire array is processed, the map entries hold the majority elements. 

%Even though this heuristic is 100\% accurate, in the absence of its main assumption no guarantees can be made. The outcome of this algorithm relies on several uncontrolled variables, such as the order our requests appeared in. However, the nature of the algorithm presents a very welcomed side effect: Elements accessed repeatedly can be evicted from the map by elements that were accessed less times but more recently. This observation reveals MEA's favoritism towards temporal locality. Furthermore, the area overhead of this algorithm implemented in hardware remains constant, regardless of how many elements need to be profiled (i.e. regardless of how many pages exist in main memory). 

In our application of MEA to activity tracking, the sequence of page addresses accessed correspond to the array of integers in the above example. 
However, we find that the sequence of accessed pages typically does not
meet the condition of MEA that guarantees it will find the most-accessed
pages; thus, it becomes an approximation.  What makes MEA most useful,
though, is its failure mode -- when it fails to find the most-accessed pages,
it does so by favoring recency over quantity.  That is, a page accessed several
times near the end of an interval can easily knock out a page accessed many
more times early in the interval.  As a result, it combines both access
counting and temporal locality, at a fraction of the cost of access counting
alone.

\remark{Nuwan's comment on next sentence: The "number of regions in memory" isn't really the input set, right? Input set is explained above as the sequence of accesses. Further, it's not entirely true that area overhead is constant with the number of regions. As the region count grows, the size of the map needed to hold those region IDs will grow. Admittedly, it grows as log2(number of regions). So it's very slow growth, but not constant. I won't edit; I'll let you address these points however you see fit.}
\remark{A.P.: I agree that the term input set is wrong here. I updated the text. But I disagree that MEA overhead grows. We could keep it at 128 if we want regardless of the memory size. It's possible that we'll want to use more counters, but the option of keeping it constant is still there}
MEA's area overhead remains constant regardless of the size of the input space (in our case the number of regions in memory). Its $O(N)$ complexity works well for analyzing a stream of access requests in real time, and eliminates the need for sorting the counters.

%In a memory management scheme that uses Full Counters and a scenario where we want to identify the 100 ``hottest'' pages, we would need one full counter per memory page and on top of that we would have to periodically sort all those counters to pick the top 100. With the use of MEA counters we only need a map with 100 entries regardless of the actual number of pages in main memory. In an 8GB memory with 2kB pages and looking for the top 100 pages, MEA needs $\sim$5K times fewer bits than the full counters' storage requirements (4MB Vs 850B). Considering all the potential benefits MEA can offer in theory, we compared its counting and prediction accuracy against the Full Counters (FC) scheme. 

\subsubsection*{MEA Evaluation}

In this section, we seek to understand the effectiveness of MEA's counting 
and prediction accuracy, compared against a Full Counters (FC) scheme,
independent of the MemPod architecture. We use memory traces captured 
from multi-programmed 8-core workloads (the same traces used and described in Section \ref{sec:Results}) and simulated MEA and FC side-by-side with an in-house off-line simulator that provides oracle knowledge of future intervals. 
\remark{Nuwan's comment on next sentence: What's the significance of 50us here?\\A.P.: At the time, 50us used to be the best interval length for MemPod so I went with it.}
The interval size for both MEA and FC was set at 5500 requests which is the average number of requests serviced within a 50us window in our timing experiments. For this experiment we used 128 MEA counters and FC requires 4.5M counters (assuming a 9GB memory capacity). 
We do two comparisons in this study.  First, we examine the ability of MEA
to identify the top pages in the past interval, something the full counter
scheme will do perfectly.  Second, we examine the ability of both schemes
to predict the top pages in the next interval.  In both studies, we will
examine the schemes' ability to identify the 30 most-accessed pages, in
bins of 10 each (1-10, 11-20, and 21-30).

\remark{These results don't make sense without knowing how many MEA
counters you used.\\A.P.: Added in text above.}

Figure \ref{fig:mea_1} shows the counting accuracy of MEA for the 
past interval, which should be compared with FC's perfect accuracy.
In some workloads, MEA identified up to 75\% of the top pages. 
However, on average MEA reports accuracy below 40\% on the top tiers. 
Thus, it is a surprisingly ineffective replacement for accurate counters,
if accurate counting were our priority.  The bias toward recent accesses
has a strong effect on the final value of the MEA counters.

\remark{I find this figure odd -- I'd rather see the results grouped by
benchmark grouping than by tier.\\A.P.: Fixed.}

\begin{figure}[t]
\centering
%  \includegraphics[width=0.46\textwidth, height=9em]{figures/mea_1_v2.pdf}
  \includegraphics[scale=.3]{figures/mea_1_v2.pdf}
  \caption{MEA counting accuracy compared to Full Counters on the top three tiers (ranks 1-10, 11-20, 21-30). Average results for homogeneous(AVG HG), mixed (AVG MIX) and all (AVG ALL) workloads shown due to space limitations.}
  \label{fig:mea_1}
\end{figure}

When we instead examine the effectiveness in identifying future hot pages,
we see a different story.  Figure \ref{fig:mea_2} presents a comparison of MEA and FC in terms of prediction accuracy. We compare each mechanism's 
``predictions'' against the top three page tiers of the following interval based on oracular knowledge. 
Using 128 counters, MEA will return \textit{up to} 128 predictions based in the past interval, while FC will return an overall ranking of each page accessed. The set of pages returned by FC with non-zero counts is guaranteed to be equal to or larger than MEA's set. To accurately compare MEA's capability against FC's, we used the same number of predictions for both schemes at each interval (i.e. The number of pages returned by MEA).
Figure \ref{fig:mea_2} plots the number of hits on predicted hot pages from the previous interval. We also selected the most interesting individual benchmarks and show them in Figure \ref{fig:mea_3} to provide a more detailed comparison. 

\remark{Here we're again missing all the critical
details of the experiment.  How many
MEA counters?  How many FC pages were used as ``predictions''?\\A.P.: 128 MEA counters. I used the number of pages MEA returned at each interval as the number of predictions for both mechanisms. Added in text above.}

On average, MEA achieves more future hits than FC by 16\%, 81\% and 68\% on the top three tiers respectively. Figure \ref{fig:mea_3} shows selected individual workloads that generated interesting results and provides a more detailed comparison. Cactus\footnote{We use a single benchmark's name as a shorthand for workloads running the same benchmark 8 times simultaneously on 8 cores.} is the only workload where FC outperformed MEA's prediction. In fact it outperformed MEA on every tier. Xalanc and mix9 are most representative of our overall
results. We can see MEA outperforming FC's prediction accuracy in every bin. 
The last two workloads we selected, bwaves and lbm, show FC failing entirely to predict the future (FC also scored zero future hits with our libquantum workload). With bwaves (and libquantum) MEA reports a very low number of future hits but not zero. These results can happen when an application streams through large structures
that exceed the size of the interval.  In that case, the past interval has 
little overlap with the next interval, but recent accesses are much more
likely to be overlapped.
Lbm shows the most interesting result of our comparison, where MEA reports one of its highest results in a workload where FC failed entirely.
\remark{Should have an explanation for this.\\A.P.: I ran the same experiment with double the epoch size to see how it behaves. Results shift but the trend remains the same. We basically get the same trend (roughly) but for tiers 4,5 and 6. Both mechanisms score zero in the top three tiers.}
Based on our lbm results it appears that 50$\mu$s is slightly larger than the benchmark's phase change period. It also seems like lbm uses a small number of pages heavily in each interval and consecutively forces FC to count those pages as hot but accesses a different set of pages in the next interval following the same behavior in all its phases. Based on the official lbm description \cite{SPEC-LBM-DESCRIPTION}, it's reasonable that lbm could have this behavior. The small set of pages accessed towards the end of the 50$\mu$s interval appears to be enough for MEA to consider those pages as hot. To verify our assumptions, we executed the same experiment with double the size of the interval (11000 requests). We observed that the trend remains the same but shifted to lower tiers. In this case both MEA and FC scored zero hits in the top three tiers. However, MEA reports high numbers in tiers 4,5 and 6, while FC still reports zero.

\begin{figure}[t]
\centering
  \includegraphics[scale=.3]{figures/mea_2_v2.pdf}
  \caption{MEA prediction accuracy (part 1) compared to Full Counters on the top three tiers (ranks 1-10, 11-20, 21-30). Results for homogeneous(WL-HG), mixed (WL-MIX) and all (WL-ALL) workloads shown due to space limitations.}
  \label{fig:mea_2}
\end{figure}

\begin{figure}[t]
\centering
  \includegraphics[scale=.45]{figures/mea_3_v2.pdf}
  \caption{MEA prediction accuracy (part 2). This graph presents the most interesting results from individual workloads.}
  \label{fig:mea_3}
\end{figure}

MEA has not previously been used in any kind of architectural event
tracking; however, our results indicate it is an attractive
alternative to full counters at a very small fraction of the hardware cost.

% !TEX root = ../HPCA2016.tex
\section{Architecture}
\label{sec:Architecture}

Our clustered migration mechanism was carefully designed to address key challenges associated with the migration problem. In this section, we present a complete description of our micro-architectural design, followed by a breakdown of all important design decisions made, along with the corresponding challenge addressed by each one.

\begin{table*}[t]
\begin{tabularx}{\textwidth}{ |X|X|X|X|X| }
  \hline
    \textbf{Challenge} & \textbf{Tradeoff} & \textbf{THM} & \textbf{HMA} & \textbf{MemPod} \\ \hline       
    Page Relocation & Flexibility / Time & Only 1 candidate \newline (Minimum / Very low)& No restrictions \newline (Max / High) & Intra-Pod migration \newline (High / Medium) \\ \hline
    Remap Table Size & Flexibility / Area & 1 entry per fast page \newline \textasciitilde2.4MB \newline (Minimum / Medium) & No remap table \newline 0 Bytes \newline (Max / Min) & 1 entry per fast page \newline 4.5MB (1.125MB/Pod) \newline (High / Medium) \\ \hline
    Activity Tracking & Accuracy / Area & 8 bits per fast page \newline 64kB \newline (Medium / Low) & 16 bits per page \newline 1.125MB \newline (Max / Max) & 48 bits per fast page \newline 384kB \newline (High / Medium)\\ \hline
    Migration Trigger & N/A & Threshold based & Interval based & Interval based\\ \hline
    Tracking Organization & Simplicity / Parallelization & Fully Centralized \newline Serialized requests \newline (High / Min) & Fully distributed \newline (High / Max) & Semi-distributed \newline Pods operate independently \newline (High / High)\\ \hline
    Migration Driver & Communication cost & CPU \newline High latency \newline (Max) & CPU (OS) \newline High latency \newline (Max) & Pod \newline Very low latency \newline (Low)\\ \hline
    Migration Cost & Time & HW cost + communication \newline (Medium) & HW + SW + cold TLBs \newline (Max) & HW cost \newline (Min)\\ \hline
\end{tabularx}
  \caption{Breakdown of state-of-the-art designs}
  \label{tab:comparison}
\end{table*}

\subsection{Clustered Migration Architecture}

 Figure \ref{fig:architecture_complete} presents an overview of MemPod. MemPod's design was kept modular to facilitate system integration. A number of memory pods are injected between the LLC and the system's MCs. Each pod clusters a number of MCs and restricts migration  within the pod. To the rest of the system, pods are seen as MCs. With MemPod's transparent design, each pod will now be receiving all the requests originally addressed to any of the pod's member MCs. 

A pod's operation when a memory request arrives would be to monitor the request, update any necessary migration-related counters and forward the request to the intended recipient MC. The migration logic within a pod does not need to be invoked during a response from any MC and could potentially be bypassed, saving some cycles. An obvious drawback of clustering MCs into pods is the serialization of potentially parallel requests to different MCs and as such, any counting scheme used by the pod -- as well as the actual forwarding of the request -- have to be as efficient as possible. 

\begin{figure}[h]
  \includegraphics[width=0.46\textwidth]{figures/dummy.pdf}
  \caption{MemPod high-level architecture}
  \label{fig:architecture_complete}
\end{figure}

\begin{figure}[h]
  \includegraphics[width=0.46\textwidth]{figures/dummy.pdf}
  \caption{Major architectural Pod elements}
  \label{fig:architecture_pod}
\end{figure}

\subsubsection*{Memory Pod}
The major architectural elements of a Pod are presented in Figure \ref{fig:architecture_pod}. A Pod's remap structure includes the remap table as well as necessary decoding and hit detection logic. Counting logic is equally important for migration as the existence of a remap table and it's responsible for identifying the hot pages which will later be candidates for migration. Without reliable hot page detection, migration would be random and often counter-productive. Finally, migration logic is responsible for orchestrating a page swap between two MCs and updating the Pod's state.

During the design of a new system, the number of pods can vary arbitrarily given different constrains. A design with just one cluster would be equivalent to a centralized migration controller allowing any-to-any\footnote{Any-to-any Migration: Page migration without limits on source and destination. All MCs can migrate a page to any MC.} migration, while a design with a pod number equal to the number of MCs would imply that migration is disabled. The latter option would be completely redundant and is only used as an example. A reasonable number of Pods would be equal to the number of slow-memory MCs. Such a configuration inherently restricts migration between slow MCs, while at the same time maintains full channel-level parallelism on the system's bottleneck -- the slow MCs. In Figure \ref{fig:architecture_complete} we present a system with eight MCs for the fast, on-die stacked memory and four MCs for the slow off-chip memory. The use of four pods imposes few restrictions on migration possibilities, while forbiding migration between two slow-memory MCs. For the remainder of this paper, we set the number of pods to four.

The following subsections provide detailed descriptions of all building blocks of a migration management mechanism. We present MemPod's, THM's and HMA's approach to each block's design, as well as alternative designs. It's important to note that each of the following building blocks is independent and future mechanisms could choose almost any element design combination in a plug-and-play fashion, with some exceptions. For example, the use of Majority Element Algorithm (MEA) activity tracking cannot work with threshold-based triggers.

\subsection{Page Relocation and Remap Table Size}
\label{sec:relocation}

Migration of memory pages can provide maximum benefits when no restrictions are imposed on the available migration locations. In other words, the optimal scenario for a migration policy would be the option of potentially filling the entire fast memory with migrated hot pages. On the other hand, more options require more bookkeeping and incur a higher cost. Flat address space memories do not have the luxury of a backing memory, like a 3D-stacked DRAM cache. Consequently, migrating a page implies swaping two pages to ensure the existence of exactly one copy for each of the participating pages. As such, migrating and swaping will be used interchangeably for the remainder of this paper.

A traditional remap table is a hash structure, indexed by a page's address and pointing to the migrated address if one exists. On a page migration, the remap table is updated to reflect the new address of a migrated pge. However, such a remap table is not enough when re-migration of pages is allowed. Figure \ref{fig:failed_remap} presents a scenario where a ``naive'' remap table fails. Figure \ref{fig:failed_remap}(i) shows the starting state of our memory before any migration, as well as the starting state of the remap table. For simplicity, we present the three memory locations needed by our example. Page 10 is assumed to be a fast memory page, while pages 100 and 200 are slow memory pages. The numbers inside the memory locations represent the content page's id. Figure \ref{fig:failed_remap}(ii) shows the state of memory and remap table after swaping pages 10 and 100. The content of page 10 is now page 100, and the content of page 100 is now 10. The remap table correctly states that requests to page 10 should be relayed to page 100 and vice versa.  Everything works as it should during this first migration, however Figure \ref{fig:failed_remap}(iii) shows the state after the second migration. Page 10 is now swaped with page 200. Such a migration would imply that page 100 (now held at 10) became cold and page 200 became hot. The contents are swaped and now page 10 holds page 200 and page 200 holds page 100. However, the state of the remap table is inconsistent. A request to page 10 would get forwarded to page 200, returning the wrong page.

This remap table design fails simply because pages are allowed to re-migrate -- like page 10 in our previous example -- while the remap table ``assumes'' the content held at a page address matches the page's ID.  There is only one solution to this problem: The migration logic needs to be aware of exactly where each page's contents are located at any given time. Such a requirement can be implemented in various ways:
%\begin{description}

	\textbf{Safe and slow:} Always restore a forwarded page's contents before it participates in a new migration. For such an implementation, hot page count will have to be kept based on the content page instead of the real page Id. In the earlier example in Figure \ref{fig:failed_remap}, the second migration of page 10 implies that page 100 is cold, but in order to offer page restoration support, the second swapping of page 10 will have to imply that page 10 is cold. The cold page (10) will be restored back to address 10 from 100 and then moved to its new location. A minor modification is required to count based on the content page's id, that does not affect any other aspect of a migration mchanism.

	\textbf{THM approach:} Migration is restricted in segments. In a memory configuration with a 1:8 fast:slow memory ratio, exactly 9 pages compete for a position at the one fast memory page available. This solution is elegant enough to allow re-migration without extremely high storage overhead, limiting however the migration potential of memory pages. If two or more hot pages coexist in a segment, only one can reside in fast memory at any given time. At the same time, if none of the segment's pages are hot, the fast memory page slot cannot be utilized by some other segment's page.

	\textbf{HMA approach:} Any page can migrate to any other page address without the need of dedicated a remap table structure. The OS-based migration scheme imposes no limitations, since the OS takes care of updating the page tables and flushing the TLB, however the cost of HMA's intervention and the penalties incurred from a cold TLB could sum up to very high values.

	\textbf{MemPod approach:} Migration is restricted within a Pod, but no intra-pod location restrictions are applied. A Pod's remap table is extended with a second field that holds the content page's id. During the second migration in our previous example, the issue was caused because we updated the remap entry of the holding page (10) instead of the entry of the content page (100). An attempt to recursively follow the remap table's entries until we figure out which one we should update risks looping infinitely because of cycles (For Example if page 10 is re-migrated back to address 10). Even if a smart algorithm is utilized to delete entries of non-migrated pages, the complexity of the recursive algorithm will only be bounded by the size of the remap table since in the worst case the entire table will be traversed. Figure \ref{fig:correct_remap} shows the memory and remap table states after the same page mirations as in Figure \ref{fig:failed_remap}. At the end of the second migration, the states of memory and remap table are consistent. It's important to note that a remap table entry now points to a pair of values: (1) \textit{relay address} (i.e. where is the requested page located) and (2) \textit{content address} (Which page is currently held).

	\textbf{Alternative approach:} Recent works propose the use of arbitrarily small remap tables. When the remap table inevitably gets full two possible solutions exist: (a) Migration is disabled or (b) the OS is invoked to update page tables and flush the TLB. After OS's intervention, the remap table is cleared and migration remains active.
%\end{description}

\TODO{Discuss the cost of MemPod and THM approach in terms of storage overhead. They should be similar.}

\begin{figure}[h]
  \includegraphics[width=0.46\textwidth]{figures/dummy.pdf}
  \caption{Naive remap table operation}
  \label{fig:failed_remap}
\end{figure}

\begin{figure}[h]
  \includegraphics[width=0.46\textwidth]{figures/dummy.pdf}
  \caption{Remap design that allows re-migration}
  \label{fig:correct_remap}
\end{figure}


\subsection{Activity Tracking Mechanism}
\label{sec:tracking}

Activity tracking could be considered the most important element of any migration mechanism. In most migration studies, activity tracking becomes a synonym of identifying hot regions by counting the number of accesses. In a more generalilzed approach, it could potentially be extended to track patterns, parallelism, bit flips or any other information useful to the underlying mechanism. Along with the remap structure, activity tracking is the limiting factor for most migration policies. The overhead of maintaining a set of counters per migration segment, is often a bottleneck. 

MemPod utilizes an important observation to maintain a low activity tracking cost. Moving the hottest pages into fast memory is a commonly accepted approach, but not necessarily optimal. Several scenarios expose the failure of such an approach. For example, a page might become cold soon after it's migrated, wasting a space in fast memory. Another example arises when interval based migration policies are used. A cold page of the previous tracking cycle could become hot during the next interval. Strong indications exist that a combination of temporal as well as spatial locality has the potential of exposing better results. 

Frequently encountered solutions in the literature consist of increasing the activity tracking granularity (i.e. track a group of pages together), limiting the bits for each tracking counter, or simply paying the overhead for a complete tracking mechanism at the finer granularity. MemPod's activity tracking  is designed with a novel approach, using a Majority Element Algorithm (MEA) to track the hottest pages at a low cost. To the best of our knowledge, such an algorithm was never used in this context. THM also presents an interesting tracking approach, by utilizing competing counters for each segment.

Using counters for every memory segment supported by the migration mechanism obviously imposes extremely high area overhead but benefits in accuracy. Identifying the hottest pages however, also requires the often-overlooked sorting complexity. With the introduction of new memory technologies and the continuous capacity increase in memory capacities, it won't be long before even the most efficient sorting algorithm will require more time than we are willing to spend.

%\begin{description}
	\textbf{THM approach:} One 8-bit competing counter tracks each memory segment. As described in Section \ref{sec:relocation}, THM restricts migration within segments. The competing counter is incremented by one when a page in slow memory is accessed and decremented by one when the segment's fast page is accessed. The counter's value is then monitored and can trigger migration. Competing counters represent a tradeoff between area overhead and accuracy. THM requires 8 bits per fast memory page making THM the most area-efficient as far as tracking is concerned. However, competing counters are susceptible to some error, since a cold page could potentially trigger migration and be placed in the fast memory.

	\textbf{HMA approach:} Full activity tracking per OS page (4KB) for all memory regions. THM uses the least efficient tracking mechanism in exchange for perfect knowledge at a fine granularity. Full activity tracking also introduces the complexity of sorting all the counters to identify hot pages. THM and MemPod do not require sorting.
	\textbf{MemPod approach:} MemPod requires one counter per hot memory page ranking close to THM's area efficiency. However, the set of MemPod's counters is capable of tracking pages \textit{from all memory regions.} using the Majority Element Algorithm presented in \TODO{cite}. For the remainder of this paper we will use the term ``MEA counters'' as a shorthand for MemPod's activity tracking. MEA is a simple, streaming algorithm which returns the (K) most frequently occuring elements in an array in linear time, using K counters. These MEA counters are organized as a hash structure, meaning we need a total of 48 bits per counter. The first 32 bits will be used for indexing, while the rest 16 will be the actual counters. 

As described in Section \ref{sec:related_work}, MEA is guaranteed to return the set of K hottest pages under certain assumptions that are not commonly held in a stream of memory requests. A sensitivity analysis of MemPod's tracking mchanism's accuracy is presented in the experimenal evaluation section. MEA strikes a balance between most frequently occuring and most recently used page addresses, a fortunate and welcomed consequence in locality exposure. As already discussed, asking for the K hottest pages does not necessarily mean that every other page is completely cold. For example, the K+1 page will usually be a good choice for migration as well. A new limitation arises when MEA counters are used: The system will be presented with the set of K hottest pages, however the counters' values cannot provide an order. The absolute hottest page could have a lower counter value than any other page.
%\end{description}

Even with the most efficient tracking mechanism, future designs will soon be required to cache some of their counters while the rest are stored in memory to aleviate the area overhead. THM's segment-based counters are automatically cached along with their corresponding SRT entry and restored whenever necessary, at the cost of a memory access. It is also important to note that any counter update should be moved off the critical path. Extreme accuracy is not necessary for correct operation. Even if the migration mechanism is not as accurate as it could be, all memory requests will be able to retrieve correct information as long as the state of memory and remap structure are consistent. Updating a remap table entry however, needs to be performed accurately without any ambiguity to aleviate the risk of inconsistent state.

With the recent growth in PIM (Processing-In-Memory, sometimes called Near-Memory Computation) mechanisms \TODO{[cite]}, future migration designs could rely on the PIM module to update the necessary counters, off the critical path and stored entirely in memory until it's time to use the actual counter values. The PIM approach \textit{eliminates} the need to store counters in SRAM circuits. Hypotheticaly, PIM could also be invoked for sorting the tracking counters.

\subsection{Migration triggers}
Deciding when to perform migration is as important as knowing which pages to migrate. Migration adds a significant delay to a system and as such it must be used wisely. Any penalties incurred should be amortised by the performance improvement when placing a page in the fast memory. Requests that arrive while migration is performed have to be delayed to ensure functionaly correct behavior. Two very common triggers are used throughout the literature whenever state must be updated based on tracking information (such as MC scheduling, NUMA, DVFS etc.). Interval-based (or epoch-based) triggers occur with a steady frequency, while threshold-based solutions trigger without a predetermined frequency, whenever a threshold value is passed. 

Both interval-based and threshold-based approaches face the same challenge of identifying the optimal interval or threshold value. Identifying the appropriate value is not usually a trivial task. Factors like a system's architecture, application's behavior, as well as semi-random factors (for example DRAM will refresh more frequently under higher temperatures) make the optimal value differ from system to system. Designers usually opt for the value that provides the best results on average. The optimal value should not be too small since it will trigger some potentially expensive procedure frequently, but it cannot be too large since that usually leads to potential performance loss. 

As far as memory migration in flat address spaces is concerned, the state-of-the-art mechanisms trigger their migration procedures based on:
%\begin{description}

	\textbf{THM approach:} THM uses a threshold-based mechanism. When the competing counter described in section \ref{sec:tracking} exceeds a threshold value, migration is triggered. THM will swap the page that triggered the event with the page currently residing in the segment's fast memory page. As a result, a small chance exists that a cold page was accessed at the right time to trigger migration and now it resides in fast memory. Such a mistake should be quickly get resolved however, since the cold page in fast memory should get remigrated soon enough. Each segment can trigger migration independently and asynchronously since no interval is used. THM risks very frequent migrations that will stall the stream of incoming requests until each swap is finished. \TODO{Check if they studied any of the previous two effects.} THM's authors identified the optimal threshold value as \TODO{XX}. 
	\textbf{HMA approach:} HMA uses an interval based mechanism. Upon each interval, HMA attempts to migrate as many pages in order to fill the fast memory. However with the high cost associated with the OS's intervention, management and the penalties of cold TLB force the interval value to be much larger. HMA authors identified the optimal timing interval to be as high as 1ms. \TODO{VERIFY}.
	\textbf{MemPod approach:} MemPod uses timing intervals. Like HMA, MemPod attempts to fill the fast memory with hot pages on each interval. However, MemPod is transparent to the system and as such the need for costly OS intervention is waived. With the cost of a migration cycle kept to lower values, MemPod offers the possibility of a smaller interval time, which could potentially result in better performance. The optimal interval value is evaluated in the results section.
%\end{description}

\subsection{Decentralization of migration logic}

The use of multiple MCs and multiple channels in modenr memory organizations serves the purpose of exposing channel-level parallelism. Each channel can issue requests independently without any knowledge of other channels' states. Some migration mechanisms in the literature inherently ``assume'' a centralized migration controller in charge of monitoring traffic, while others attempt to implement a completely distributed mechanism. A centralized approach can be severely limiting. Current HBM memory technology allows up to 8 channels, while many processors are already designed with two or even more off-chip memory channels. Our evaluated system in this paper features a total of twelve memory channels. Channel number is predicted to increase in the near future \TODO{[cite]}. The channel parallelism capabilities will be entirely lost if we enforce request serialization due to migration-related activity tracking or remap table lookups. On the other hand, a fully distributed solution will eliminate all serialization, at the cost of all-to-all communication between each channel. An alternative to the communication cost would require OS intervention. MemPod's novel clustered architecture attempts to balance this tradeoff. 

As with most of the essential elements for migration presented in this Section, the system's designer can choose any level of centralization desired according to the specific design's constraints. As such, the body of work on migration covers the entire range:
%\begin{description}

	\textbf{THM approach:} Even though not clearly stated in the THM proposal, it appears the authors opted for a centralized unit and consequently all channel parallelism potential is lost, placing THM last in our list in terms of parallelism potential. Decentralizing THM's migration controller appears to be possible due to the possibility of caching SRT entries, but in that case cache coherency becomes a concern. Race conditions could occur if the same SRT entry is simultaneously cached in different locations and its counter is modified. For this paper's evaluation section, we assume THM is fully centralized, as presented through figures in its proposal publication.

	\textbf{HMA approach:} HMA ranks at the top of our list, featuring a fully decentralized mechanism. It's important to note that HMA does not require a remap table and consequently one possible source of request serialization is automatically removed. Activity tracking is performed at each MC individually, where a controller will monitor the activity of its own pages. When HMA's migration is triggered, the OS will collect all activity monitors from all the MCs before it proceeds with migration. Of course, collecting all this information from each MC by the OS consumes a considerable amount of cycles.

	\textbf{MemPod approach:} Clustering memory channels into Pods comes with significant benefits. First, assigning exactly one off-chip (slow) memory channel to each independent Pod ensures those channels can still issue requests in parallel. Being the slowest part of our memory hierarchy, keeping slow channels independent does not add delay to a potential bottleneck. Furthermore, each group of two on-chip (fast) channels can still operate in parallel. Some serialization is introduced between sibling fast channels, however with a Pod's light design and the high-bandwidth potential of those channels, the delay is amortized. Beyond channel parallelism, each Pod can hold all the migration-related structures, eliminating the need of retrieving information from each of its MCs at the beginning of a migration interval. \TODO{I'm trying to point out that a Pod keeps the structure size manageable because it handles less controllers. Other policies could hold all the information at a central point, but structures for 12 channels will be a log bigger thus slower. I think I'm not presenting my point correctly here.}
%\end{description}

\subsection{Migration Datapath}
Regardless of the choice for each migration building block described so far, once migration is triggered, any migration manager has to follow the same steps: First, migration candidates need to be identified. Traditionally, one page (or a segment depending on the migration granularity) from the slow memory and one from fast memory. The two identified candidates need to be swapped. First they will be read and stored in temporary buffers and then written at their remapped locations. 

Describing the actual migration datapath is often overlooked in migration publications. Without dedicated page migration driver hardware, migration will have to be orchestrated by some CPUs. Consequences include communication delay, potentially some delay introduced at the processor's cache levels and the performance degradation caused by stalling those CPUs until migration is over. MemPod implements the migration driver within each Pod. Since the Pod has direct communication with the MCs, added delays are kept to a minimum. In HMA, the OS orchestrates everything. Some CPUs have to be stalled and used to service the OS interrupt, causing the migrated pages to traverse through communication mediums and caches on each way. THM does not describe the datapath in detail. We assume the CPUs are used in this case too.

We assume that channel parallelism is utilized when reading and writing the candidate pages. In other words, the two read commands will be sent in parallel, as well as the write commands that follow. Consequently, The hardware penalty for one page swap is the time required to read from and then write to the slow memory. We also assume that writing the two candidates back occurs a row buffer hit since the page was just opened in the previous step. When consequtive swaps happen, as in the case of HMA and MemPod, all swap reads are assumed to result in row buffer misses. In this study, we evaluate all mechanisms under the same memory organization and as such, the hardware swap penalty is the same regardless of the mechanism. However, each mechanism introduces some unique penalties:
%\begin{description}

	\textbf{THM} does not introduce a cost for identifying the candidate pages since it follows a deterministic algorithm. The page that caused the competing counter to exceed the set threshold will be the slow memory candidate, while there is exactly one fast page candidate per segment. Furthermore, only one swap is executed at every triggered migration, setting te cost per migration equal to the hardware cost.
	
    \textbf{HMA } attempts to fill the entire fast memory with migrated pages. The number of swaps that will be performed per interval can be as high as the number of pages in fast memory. Inevitably some hot pages will already reside in the fast memory. We modeled HMA to not attempt migration of those hot pages. Such an approach could compicate finding a fast-memory candidate page, although with HMA's full activity tracking counters, and since sorting is a necessary operation at every interval, this problem can be reduced to the simple task of following the sorted activity list backwards (i.e. It's easy to find the coldest fast-memory page). Unfortunately, HMA introduces costs that are hard to estimate. Traversing and updating page tables and flushing TLBs is part of the cost introduced by the OS. On top of that, the effect of a cold TLB can penalize severely all running applications.
	
    \textbf{In MemPod,} similar to HMA, each Pod tries to fill the entire fast memory assigned to it. With the use of MEA counters, identifying the fast-memory page candidate is as simple as checking that it's not part of the N hot pages. The identification algorithm starts at the very first fast memory location and iterates sequentially until it detects a page address that is not in the set of hottest pages. For the next migration, the identification algorithm simply continues from where it was left. If a hot page already resides in the fast memory it's ignored. Since we have exactly N MEA counters, we guarantee that the fast memory will have enough pages to accommodate all hot pages.
%\end{description}

\TODO{Give numbers for hardware migration penalty}

\subsection{Scalability to Future Memory Capacities}
Assuming memory capacity in the order of tens of Terabytes, SRAM requirements for activity tracking and remap tables could become unfeasible. Caching part of the migration logic and using part of main memory as backing storage seems necessary. An analysis on the impact of such caching, as well as the optimal cache size is presented in the experimental evaluation section. MemPod's semi-distributed architecture allows caching without any action required to protect against race conditions because of the utilization of independent Pods that never share information.

At such capacity levels, centralized migration controllers will no longer be sustainable due to the severity of the introduced serialization penalty. Assuming the driving force behind memory capacity's tremendous scaling are future applications, it's safe to assume that memory traffic would also scale up, as well as sustainable bandwidth expectations. 

Alternative approaches described in this Section limit the size of the remap table at the cost of disabling migration when full, or invoking the OS before migration resumes. This approach could also be incorporated in future migration mechanisms. We should note that with a limited remap table size, or even an available number of counters less than the number of fast pages, MemPod would still be able to migrate \textit{any} slow memory page to the limited set of fast pages available for migration, in contrast to THM, where only those segments that happened to fit in the new limited size will be able to migrate, forbidding migration of the rest of the slow pages. HMA would remain intact from this limitation since no remap tables are necessary.

\subsection{Discussion and comparison}
\TODO{Talk about limits in the size of remap tables/counters just fr the sake of argument. How would each mechanism handle it?}




















































% !TEX root = ../MemPod.tex
\section{Results}
\label{sec:Results}

\subsection{Evaluation Framework}
\label{sub:Evaluation}

The goal of our evaluation framework is to quantitatively and qualitatively assess MemPod's capabilities and compare it against state-of-the-art proposed mechanisms. Throughout our evaluation section, we study MemPod's performance running as part of an eight-core CPU. We extended Ramulator \cite{kim-ramulator} to support flat address space hybrid memories and with MemPod for our memory simulations. HMA and THM were also implemented in our simulation framework for comparison purposes. Ramulator allows for cycle-level memory simulation and includes a simple CPU front-end capable of approximating resource-induced stalls. We chose to evaluate MemPod under a realistic memory configuration consisting of 1GB 3D-stacked HBM2.0 \TODO{[Cite]} and 8GB of off-chip DDR4-1600. Table \ref{tab:specs} Provides a more detailed description of the simulated system's configuration.

\subsection{Experimental Methodology}
\label{sub:Experimental}

We used benchmarks from the SPEC2006 suite \cite{spec} as our workloads. Using Sniper \cite{sniper}, we extracted memory request traces while simultaneously executing 8 benchmarks on a simulated 8-core CPU. We then feed these multi-programmed memory traces in Ramulator, executing all workloads to completion. Our complete set of workloads consists of 16 ``homogeneous'' workloads, where the same benchmark runs 8 times in parallel (we simply call this workload with the benchmark's name in later results), as well as 12 workloads featuring a random mix of 8 benchmarks each (marked as mix1-12). A breakdown of the mixed workloads is shown in Table \ref{tab:workloads}.

We also extended Ramulator with caches needed for activity tracking and/or remap tables depending on the simulated mechanism. Cache misses inject a memory request into the stream of requests fed by our trace files to retrieve the missing information. No priority is given to cache miss requests over the rest of the requests. When caches are disabled, the simulator assumes that any information needed by any mechanism exists on chip and is accessible without any delay. The migration process was implemented in detail as well. In order to read an entire page from memory, 32 read requests need to be sent for each of the two migration candidates and then another set of 32 requests for each of the two write-backs.

Since we used Ramulator with recorder traces, we chose to report Average Main Memory Time in our results instead of IPC. Even though Ramulator has the ability to approximate IPC, AMMT is a more accurate metric since it models the memory in detail. AMMT is the average time spent in main memory by each request (lower is better). Due to space limitation we are not able to show results from all our workloads in most of the graphs in this paper. In those graphs, we only present the results from mixed workloads, the average of all mixed workloads, average of all homogeneous workloads and overall average.

\begin{table}[t]
  \includegraphics[width=0.46\textwidth]{figures/specs_table.pdf}
  \caption{Experimental framework configuration}
  \label{tab:specs}
\end{table}

\begin{table}[t]
  \includegraphics[width=0.46\textwidth]{figures/workloads_checkmarks.pdf}
  \caption{Mixed workloads description}
  \label{tab:workloads}
\end{table}
%
%\begin{table}
%  \includegraphics[width=0.46\textwidth]{figures/workload_characterization.pdf}
%  \caption{Experimental framework configuration}
%  \label{fig:specs}
%\end{table}

\subsection{Simulation Results}
\label{sub:SimResults}

\subsubsection{Optimal Parameter Values}

MemPod exposes 3 variables that allow fine-tuning it based on a particular system or expected workloads: (1) The number of MEA counters, (2) interval length and (3) the size of each MEA counter. Identifying the optimal values for each parameter can maximize MemPod's capabilities. The number of MEA counters dictates the highest possible number of migrations that can be performed at each interval, while the epoch length will determine MemPod's ability to better adapt to phase changes in a workload. The size of each MEA counter can affect performance due to the loss of information when smaller counters are used but can also save space on the chip.

We first identified the optimal number of MEA counters, by keeping the epoch length constant and exponentially increasing the number of counters from 16 to 512. In order to minimize the impact of other factors, we executed this experiment with 16 bits per counter and caches disabled. In other words, each counter was given more than enough space and all required information such as the remap table resides entirely on the chip and is accessible at no cost. 

Figure \ref{fig:num_counters} shows normalize Average Main Memory Time (AMMT)\footnote{AMMT: The average time a request spends in main memory}, along with the average number of migrations per Pod per epoch (secondary axis). The results indicate that each Pod utilizes the higher number of counters and consecutively performs more migrations per interval, however performance begins leveling off when more than 128 counters were used. More migrations can directly be translated into higher power consumption and communication cost. Using 128 counter, MemPod improves AMMT by 7\% over the baseline with 16 MEA counters. Based on our observations we conclude that the optimal value for this parameter is 128 and will be used for the remainder of this section.

\begin{figure}[h]
	\centering
  \includegraphics[width=0.46\textwidth]{figures/avg_num_counters_normalized.pdf}
  \caption{\# of MEA Counters Vs Normalized AMMT (primary axis) and average \# of Migrations per Pod per interval (secondary axis)}
  \label{fig:num_counters}
\end{figure}

After identifying the optimal counter number, Figure \ref{fig:interval} displays the same measurements as Figure \ref{fig:num_counters} with a varying epoch length. Caches were again disabled, each counter was given 16 bits and the number of MEA counters was set to the optimal value of 128. In most of our benchmarks a smaller epoch length leads to higher performance. In accordance to the previous experiment, higher epoch lengths also lead to higher number of migrations. On average, an epoch length of 100us outperforms the baseline (50us) by 2\%. For comparison purposes, HMA \cite{meswani-HPCA21} identified the optimal epoch length to be 1ms (10x larger) in order to support all lengthy processes that take place during a migration event.

\begin{figure}[h]
  \includegraphics[width=0.46\textwidth]{figures/interval_length_normalized.pdf}
  \caption{Interval Length Vs Normalized AMMT (primary axis) and average \# of Migrations per Pod per interval (secondary axis)}
  \label{fig:interval}
\end{figure}

As previously explained in Section \ref{sec:MEA}, the MEA algorithm cannot be cached efficiently and as a result, the entire activity tracking structure needs to be on chip. The size (in bits) of each counter defines the area requirements of our MEA tracking mechanism. We modified the MEA algorithm to remove map entries with counter values equal to or less than zero (instead of strictly equal to zero) in order to support counter saturation. We opted not to immediately remove an overflowed counter even though its value is now zero, since the existence of the correct counter set is crucial to the algorithm's accuracy. For this experiment we used the optimal parameter values identified in the previous experiments and set the number of MEA counters to 128 and interval length to 100us. All caching was disabled in order to study the direct impact of this variable.

Figure \ref{fig:counter_size} presents the impact of counter size on AMMT and average number of migrations. We first observe that 8 bits are sufficient for the majority of our workloads, since larger sizes report identical results. Our second observation is that two bit counters report a negligible performance degradation (0.3\% on average) and a reduced average number of migrations.

\begin{figure}[h]
  \includegraphics[width=0.46\textwidth]{figures/counter_size_normalized.pdf}
  \caption{Counter size (in bits) Vs Normalized AMMT (primary axis) and average \# of Migrations per Pod per interval (secondary axis)}
  \label{fig:counter_size}
\end{figure}

The most interesting observation comes from assigning 4 bits to each counter. On average, performance is boosted slightly (up to 3\% and 1\% on average) even compared to using larger counters, while migration count is smaller. Since a difference is observed, we can conclude that these smaller counters ``lose'' information that in turn benefits overall execution. The reported result is a welcomed artifact of the MEA algorithm combined with application behavior. 

Based on our results, we identify 4 bits per counter to be the optimal value. Each one of the 128 MEA entries needs 21 bits for addressing the 1125K pages per Pod and 4 bits for its counter, leading to an area cost of only 400B per Pod and $\sim$1.5KB total. Compared to the state of the art, MemPod's activity tracking requirement is $\sim$341x smaller than THM's (512KB) and $\sim$6100x smaller than HMA's (9MB).


\subsubsection{Performance Comparison}
\label{sub:performance}

\begin{figure*}[t]
  \includegraphics[width=\textwidth]{figures/performance_over_nlm.pdf}
  \caption{Performance Comparison: AMMT is normalized to a hybrid memory without any migration mechanism.}
  \label{fig:performance}
\end{figure*}

Figure \ref{fig:performance} presents a performance comparison of MemPod, HMA, THM and a configuration with 9GBs of on-chip HBM memory, normalized to the performance of a hybrid memory configuration without migration capabilities. We evaluated all mechanisms with caching disabled. 

Based on the results we can derive some interesting observations:
\begin{itemize}
	\item In some workloads migration is harmful to performance, as observed with the bwaves workload, where a no-migration scheme reports higher performance (lower AMMT). We observe that in those cases, MemPod leads to deteriorated performance compared to THM. However, in the case of zeusmp, MemPod increases performance, while THM and HMA report higher AMMT than the no-migration scheme.
	\item MemPod outperforms the state-of-the-art competitors in the majority of our workloads, and in several cases scoring very close to an HBM2-only configuration. 
	\item On average MemPod reports 25\% higher AMMT than HBM2-only, while THM and HMA report 39\% and 41\% respectively.
	\item All mechanisms outperform HBM2-only when executing the libquantum experiment. We attribute this result to a combination of correct timing, application behavior and workload size. In the case of libquantum, to working set size fits entirely in our fast memory. As a results after some migrations, the entire working set will be present in our HBM. However, correct timing is the driving factor behind this impressive performance increase. Our results show the row-buffer hit ratio to be ??x times larger than having HBM2-only (and random page assignment). Apparently, page migrations resulted in an in-memory page order that exploits almost every bit of memory parallelism from the application. This result could be further explored and intentionally recreated in some future work.
\end{itemize}

\subsubsection{Caching Effect}

\begin{figure}
  \includegraphics[width=0.46\textwidth]{figures/cache_impact.pdf}
  \caption{\small{Cache Impact}}
  \label{fig:cache}
\end{figure}

Migration mechanisms will be forced to include a cache since activity tracking and remap table structures are commonly too large to hold on-chip. The use of a cache will unavoidably hinder performance. In this experiment we evaluate the impact of a cache on each mechanism's performance. As described in Section \ref{sec:Architecture}, each mechanism has different cache requirements. THM caches its counters and remap table together with its ``Segmented Remap Table'' structure. HMA has no need for a remap table however it has high storage requirements for its counting mechanism. MemPod only needs to cache its large remap table since MEA counters will be on chip. We simulated each mechanism with 16, 32 and 64 KB of cache. For MemPod, the available cache capacity was divided equally over four Pods.

HMA's design further complicates this study, since sorting all activity counters at each epoch is performed by the OS, utilizing the cpu's cache instead of the dedicated migration cache. We present three flavors of HMA in this experiment: HMA-OPT does not take into account any penalties for OS interrupts, TLB shootdowns or Page Table (PT) updates, walks and misses. HMA-0.01 assumes a 1\% PT miss rate when updating the PT after each migration (not during program execution since that would have led to an unfair comparison) and HMA-0.05 assumes a 5\% miss rate. Both HMA-0.01 and HMA-0.05 take into account the penalty for the OS interrupt and the TLB shootdown. Even though sorting all HMA's counters is a costly procedure, in this experiment we assume a ``best-case scenario'' where the entire sorting process is overlapped by requests being serviced from the memory and we don't penalize HMA for sorting. We also do not model the application-level effects of starting with a cold TLB after each interval. Based on HMA's proposal \cite{meswani-HPCA21}, we model its penalties as: 3$\mu$s per TLB shootdown, 2$\mu$s per OS interrupt and 5$\mu$s per PT miss.

Figure \ref{fig:cache} shows our results obtained by taking the average AMMT from all our workloads for each mechanism. HMA-OPT reports the lowest AMMT, albeit unrealistic and MemPod outperforms every other mechanism regardless of the cache size. MemPod outperforms THM by 9\%, and HMA-0.01/0.05 by 6\% and 40\% respectively with a 64kB cache, while HMA-OPT outperforms MemPod by 9\%.

\subsubsection{Scalability}

We simulated all mechanism under an overclocked memory configuration in order to evaluate how each one scales with memory technology advances. For this experiment, we used a 4GHz HBM as our stacked memory and a DDR4-2400 as our off-chip memory. Figure ?? shows our results. As memories become faster, MemPod increases the performance gap over the state-of-the-art mechanisms. With this memory configuration, MemPod outperforms THM by ??\%, and HMA-0.01/0.05 by ??\% and ??\% respectively, while HMA-OPT outperforms MemPod by ??\%.
% !TEX root = ../HPCA2016.tex
\section{Conclusions and Future Work}
\label{sec:Conclusions}

MemPod is a scalable, modular and efficient dynamic memory management mechanism. Our analysis demonstrated significant and encouraging results compared to state-of-the-art proposals. The modular design achieved with the use of Pods allows for a more scalable migration mechanism while at the same time enforcing small limitations on migration opportunities.

MEA counters have not been previously used in the micro-architectural context. Our analysis demonstrates it can be beneficial in more than just a migration mechanism. Any proposed mechanism that needs some sort of activity tracking or even identifying frequently-occuring phenomena could possibly gain significant performance benefit at lower cost through the use of MEA.

\TODO{Not sure what to keep/remove here}. I'll think a little more about it.}
In addition, MemPod leaves a lot to be investigated in future work. We intend to focus on extending our design to support even more complex memory configurations, with the addition of non-volatile memory which adds a reliability aspect. Ordering of migrations could result to incredible increase in performance, as demonstrated in experiment ?? and libquantum, allowing for future research. MemPod could also be used to accommodate memory scheduling mechanisms by utilizing its internal structures, or on the other hand, we could focus on ``Pod-aware'' memory scheduling.

%\bstctlcite{bstctl:etal, bstctl:nodash, bstctl:simpurl}
%\bibliographystyle{IEEEtranS}
\bibliographystyle{ieeetr}
\bibliography{references}

\end{document}
