\usepackage{ifthen}
\usepackage{calc}
\usepackage{pifont}
\usepackage{color}
\usepackage{fancyhdr}

\newcounter{hours}
\newcounter{minutes}
\newcommand{\printtime}{%
  \setcounter{hours}{\time/60}%
  \setcounter{minutes}{\time-\value{hours}*60}%
  \thehours:\theminutes}

\newcommand{\myspacing}{1.62}
\newcommand{\linespacing}[1]{\renewcommand{\baselinestretch}{#1}\normalsize}

% Set timeofmake to 'true' to print the time the file was compiled
\newboolean{timeofmake}
\setboolean{timeofmake}{false}


\newcommand{\todo}[1]{{\color{red}\sf\bfseries #1}}
\newcommand{\dontinclude}[1]{ }
\newcommand{\myfootnote}[1]{\footnote{\renewcommand{\baselinestretch}{1.0}\scriptsize #1}}


\newcommand{\putsec}[2]{\vspace{-0.1in}\section{#2}\label{sec:#1}\vspace{-0.0in}}
\newcommand{\putssec}[2]{\vspace{-0.05in}\subsection{#2}\label{ssec:#1}\vspace{-0.0in}}
\newcommand{\putsssec}[2]{\vspace{0.0in}\subsubsection{#2}\label{sssec:#1}\vspace{0.0in}}
\newcommand{\putsssecX}[1]{\vspace{0.0in}\subsubsection*{#1}\vspace{0.0in}}

%\renewcommand{\figurename}{\renewcommand{\baselinestretch}{1.0}\vspace{-0.0in}\footnotesize Figure}
%\renewcommand{\tablename}{\renewcommand{\baselinestretch}{1.0}\vspace{-0.0in}\footnotesize TABLE}
 
\newcommand{\tabput}[3]{
\begin{table}[t]
\renewcommand{\baselinestretch}{0.95}
%\small %also for forcing a baselinestretch update
\begin{center}
{
#2
\vspace{-0.1in}
}
\caption{\footnotesize #3 \label{tab:#1}\vspace{-0.2in}}
\renewcommand{\baselinestretch}{\myspacing}\normalsize
\end{center}
\end{table}
}
  
\newcommand{\tabputW}[3]{
\begin{table*}[h]
\begin{center}
{
#2
}
\end{center}
\caption{\footnotesize #3 \label{tab:#1}}
\end{table*}
}

\newcommand{\figput}[4][1.0\linewidth]{
\begin{figure}[t]
\begin{minipage}{\linewidth}
\footnotesize %also for forcing a baselinestretch update
\begin{center}
\includegraphics[trim=#3, clip, width=#1]{plots/#2}
\end{center}
\vspace{-0.1in}
\caption{\footnotesize #4 \label{fig:#2} \vspace{-0.0in}}
\end{minipage}
\end{figure}
}


\newcommand{\figputW}[4][1.0\linewidth]{
\begin{figure*}
\begin{minipage}{\linewidth}
\footnotesize %also for forcing a baselinestretch update
\begin{center}
\includegraphics[trim=#3, clip, width=#1]{plots/#2}
\end{center}
\vspace{-0.1in}
\caption{\footnotesize #4 \label{fig:#2}}
\end{minipage}
\end{figure*}
}

\newcommand{\figputT}[3]{
\begin{figure}[t]
\begin{minipage}{\linewidth}
\footnotesize %also for forcing a baselinestretch update
\begin{center}
\includegraphics[trim=#2, clip, width=1.0\linewidth]{plots/#1}
\end{center}
\vspace{-0.1in}
\caption{\footnotesize #3 \label{fig:#1}}
\end{minipage}
\end{figure}
}

\newcommand{\figputJ}[4][1.0\linewidth]{
\begin{figure}[htb]
\begin{minipage}{\linewidth}
\footnotesize %also for forcing a baselinestretch update
\begin{center}
\includegraphics[trim=#3, clip, width=#1]{plots/#2}
\end{center}
\caption{\footnotesize #4 \label{fig:#2}}
\end{minipage}
\end{figure}
}

\newcommand{\figref}[1]{Figure~\ref{fig:#1}}
\newcommand{\tabref}[1]{Table~\ref{tab:#1}}
\newcommand{\secref}[1]{Section~\ref{sec:#1}}
\newcommand{\ssecref}[1]{Section~\ref{ssec:#1}}
\newcommand{\sssecref}[1]{Section~\ref{sssec:#1}}

\newcommand{\ignore}[1]{}


