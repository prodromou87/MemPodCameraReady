\documentclass[pageno]{jpaper}

%replace XXX with the submission number you are given from the HPCA submission site.
\newcommand{\hpcasubmissionnumber}{XXX}

\usepackage[normalem]{ulem}

\begin{document}

\title{
Guidelines for Submissions to HPCA 2014}

\date{}
\maketitle

\thispagestyle{empty}

\begin{abstract}
This document is intended to serve as a sample for submissions to HPCA 2014.
We provide some guidelines that authors should follow when submitting papers to
the conference.
\end{abstract}

\section{Introduction}

This document provides the formatting instructions for submissions to the 20th
Annual IEEE International Symposium on High Performance Computer Architecture,
2014. In an effort to respect the efforts of reviewers and in
the interest of fairness to all prospective authors, we request that all
submissions to HPCA 20 follow the formatting and submission rules detailed
below.  Submissions that violate these instructions may not be
reviewed, at the discretion of the program chair, in order to maintain a review 
process that is fair to all potential authors.  This is a generous format,
with plenty of space -- there should be no need to tweak it in any
significant way.

An example submission (formatted using the HPCA-20 submission format) that
contains the submission and formatting guidelines can be downloaded from
the location cited below.  Complete details can be found on the
conference website at {\em hpcaconf.org}.

\noindent{\bf Changes to last year's format:}

\begin{itemize}
\item Papers are limited to 11 pages not including references.
\item Each reference must specify all authors (no {\it et al.}).
\end{itemize}

\noindent{\bf Acceptance philosophy}

Every effort will be made to judge a paper on its own merits. There will be no target acceptance rate. We expect to accept a wide range of papers with appropriate expectations for evaluation.
While papers that build on significant past work with strong evaluations are valuable, papers that open new areas with less rigorous evaluation are even more so.

All questions regarding paper formatting and submission should be directed to
the \href{mailto:enright@eecg.toronto.edu}{program chair}.

\section{Preparation Instructions}

\subsection{Paper Formatting}

All submissions should contain a maximum of 11 pages of single-spaced
two-column text and figures excluding references. You can use an unlimited number of extra pages for references.
If you are using \LaTeX~\cite{lamport94}
to typeset your paper, then we
strongly suggest
that you use the template available at
\url{http://www.eecg.toronto.edu/~enright/hpca20template.tar.gz} -- this
document was prepared with that template.  This document can
also be downloaded at \url{http://www.eecg.toronto.edu/~enright/hpca20sample.pdf}.
If you are using a different
software package to typeset your paper, then please adhere to the guidelines
given in Table~\ref{table:formatting}.

\begin{table}[h!]
  \centering
  \begin{tabular}{|l|l|}
    \hline
    \textbf{Field} & \textbf{Value}\\
    \hline
    \hline
    Page limit & {\bf 11} pages, \\
& {\bf not including references}\\
    \hline
    Paper size & US Letter 8.5in $\times$ 11in\\
    \hline
    Top margin & 1in\\
    \hline
    Bottom margin & 1in\\
    \hline
    Left margin & 0.75in\\
    \hline
    Right margin & 0.75in\\
    \hline
    Separation between columns & 0.25in\\
    \hline
    Body font & 10pt\\
    \hline
    Abstract font & 10pt, italicized\\
    \hline
    Section heading font & 12pt, bold\\
    \hline
    Subsection heading font & 10pt, bold\\
    \hline
    Caption font & 9pt, bold\\
    \hline
    References & 8pt, no page limit\\
& list all authors' names \\
    \hline
  \end{tabular}
  \caption{Formatting guidelines for submission.}
  \label{table:formatting}
\end{table}

\textbf{Please ensure that you include page numbers with your
submission}. This makes it easier for the reviewers to refer to
different parts of your paper when they provide comments.

Also, please ensure that your submission has a banner at the top of
the title page, similar to this one, which contains the submission
number and the notice of confidentiality.  If using the template,
just replace XXX in the template with the submission number
you receive from the submission website.

\subsection{Content}

\noindent\textbf{\sout{Author List.}} All submissions are double
blind. Therefore, please do not include any author names in the
submission. You must also ensure that the metadata included in the
PDF does not give away the authors. If you are improving upon your
prior work, refer to your prior work in the third person.

\noindent\textbf{Figures and Tables.} Ensure that the figures and
tables are legible.  Please also ensure that you refer to your
figures in the main text. The proceedings will be printed in
gray-scale, and many reviewers print the papers in
gray-scale. Therefore, if you must use colors for your figures, ensure
that the different colors are highly distinguishable in gray-scale.
If a figure is not easily understandable in gray-scale, then assume
it will not be understood by the reviewers.  In many cases, it
is better to just prepare your documents without color.

\noindent\textbf{Main Body.} Avoid bad page or column breaks in
your main text, i.e., last line of a paragraph at the top of a
column or first line of a paragraph at the end of a column. If you
begin a new section or sub-section near the end of a column,
ensure that you have at least 2 lines of body text on the same
column.

\noindent\textbf{References.} There is no length limit for references.  {\bf Each reference must explicitly list all authors of the paper} (no {\it et al.}).  Author of NSF proposals should be familiar with this requirement.  Knowing all authors of related work will help find the best reviewers.

\section{Submission Instructions}

\subsection{Paper Authors}

Declare all the authors of the paper upfront. Addition/removal of authors once
the paper is accepted will have to be approved by the program chair.  The
paper selection process is carefully run in a way that maximizes fairness
by seeking to eliminate all conflicts of interest.  Late changes to author
lists can invalidate that process.

\subsection{Declaring Conflicts of Interest}

The authors must register all their conflicts into the paper submission site.
Conflicts are needed to resolve assignment of reviewers. Please get the conflicts
right.  You have several days between the registration of the paper and final
submission -- there is no need to do the conflicts in a rush at the last
second. If a paper is found to
have an undeclared conflict that causes a problem, the paper may be rejected.

Please declare a conflict of interest with the following for any author of a paper:

\begin{enumerate}
\item PhD advisor
\item Other past or current advisors
\item Current or past students
\item People whom you have collaborated in the last 5 years. This collaboration can consist of a joint research or development project, a joint paper, or when there is direct funding from the potential reviewer (as opposed to company funding) to an author of the paper. Co-participation in professional activities, such as tutorials or studies, is not cause for conflict. When in doubt, the author should check with the PC chair.
\item People with the same current affiliation or who were in the same institution in the last 5 years.
%\item People whom you co-authored papers with in the last 5 years
%\item People whom you co-authored funded proposals with in the last 5 years.
\item Between people whose relationship prevents the reviewer from being objective in his/her assessment. Please be reasonable.  For
example, just because a reviewer works on similar topics as the paper you are
submitting is on, you cannot declare a conflict of interest with them.
\end{enumerate}

All conflicts must be justified.
You will have to declare all conflicts with PC members as well as non-PC members
with whom you
have a conflict of interest.  When in doubt, contact the \href{mailto:enright@eecg.toronto.edu}{program chair}.

\subsection{Concurrent Submissions and Resubmissions of Already Published Papers}

By submitting a manuscript to HPCA-20, the authors guarantee that the
manuscript has not been previously published or accepted for publication in a
substantially similar form in any conference or journal. The authors also
guarantee that no paper which contains significant overlap with the
contributions of the submitted paper is under review to any other conference or
journal or workshop, or will be submitted to one of them during the HPCA-20
review period. Violation of any of these conditions will lead to rejection.

Extended versions of papers accepted to IEEE Computer Architecture Letters can
be submitted to HPCA-20.  If you are in doubt, contact the \href{mailto:enright@eecg.toronto.edu}{program chair}.

\subsection{Submission Site}

The submission site is located at \url{http://hpca20.eecg.toronto.edu/conf}.

\bstctlcite{bstctl:etal, bstctl:nodash, bstctl:simpurl}
\bibliographystyle{IEEEtranS}
\bibliography{references}

\end{document}
